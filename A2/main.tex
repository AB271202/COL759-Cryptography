\documentclass[a4paper, 11pt]{article}
\usepackage{comment} 
\usepackage{lipsum} %This package just generates Lorem Ipsum filler text. 
\usepackage{fullpage} % changes the margin
\usepackage[a4paper, total={7in, 10in}]{geometry}
\usepackage[fleqn]{amsmath}
\usepackage{amssymb,amsthm}  % assumes amsmath package installed
\newtheorem{theorem}{Theorem}
\newtheorem{corollary}{Corollary}
\usepackage{graphicx}
\usepackage{tikz}
\usetikzlibrary{arrows}
\usepackage{verbatim}
\usepackage[numbered]{mcode}
\usepackage{float}
\usepackage{tikz}
    \usetikzlibrary{shapes,arrows}
    \usetikzlibrary{arrows,calc,positioning}

    \tikzset{
        block/.style = {draw, rectangle,
            minimum height=1cm,
            minimum width=1.5cm},
        input/.style = {coordinate,node distance=1cm},
        output/.style = {coordinate,node distance=4cm},
        arrow/.style={draw, -latex,node distance=2cm},
        pinstyle/.style = {pin edge={latex-, black,node distance=2cm}},
        sum/.style = {draw, circle, node distance=1cm},
    }
\usepackage{xcolor}
\usepackage{mdframed}
\usepackage[shortlabels]{enumitem}
\usepackage{indentfirst}
\usepackage{hyperref}
\usepackage[capitalize, nameinlink]{cleveref}
\renewcommand{\thesubsection}{\thesection.\alph{subsection}}

\newenvironment{problem}[2][Problem]
    { \begin{mdframed}[backgroundcolor=gray!20] \textbf{#1 #2} \\}
    {  \end{mdframed}}

\newenvironment{reduction}
    { \begin{mdframed}[backgroundcolor=blue!20] \\}
    {  \end{mdframed}}
% Define solution environment

\newcommand{\hr}{\noindent\rule{7in}{2.8pt}}
\newenvironment{solution}
    {\textit{Solution:}}
    {\clearpage}
\newcommand{\prob}[1]{\begin{mdframed}[backgroundcolor=gray!20] \textbf{Problem #1}\end{mdframed}}
\renewcommand{\qed}{\quad\qedsymbol}
\newcommand{\bit}{\left\{0, 1\right\}}
\newcommand{\enc}{\mathsf{Enc}}
\newcommand{\dec}{\mathsf{Dec}}
\newcommand{\negl}{\mathsf{negl}}
\newcommand{\prf}{\mathsf{PRFAdv}}
\newcommand{\prg}{\mathsf{PRGAdv}}
\newcommand{\poly}{\mathsf{poly}}
\newcommand{\sign}{\mathsf{Sign}}
\newcommand{\verify}{\mathsf{Verify}}
\newcommand{\macvq}{\mathsf{MAC^{vq}}}
\newcommand{\mac}{\mathsf{MAC}}
\newcommand{\AES}{\mathsf{AES}}
\newcommand{\N}{\mathbb{N}}
\newcommand{\R}{\mathbb{R}}
\newcommand{\Z}{\mathbb{Z}}

\newcommand{\calA}{\mathcal{A}}
\newcommand{\calB}{\mathcal{B}}
\newcommand{\calC}{\mathcal{C}}
\newcommand{\calE}{\mathcal{E}}
\newcommand{\calF}{\mathcal{F}}
\newcommand{\calG}{\mathcal{G}}
\newcommand{\calH}{\mathcal{H}}
\newcommand{\calK}{\mathcal{K}}
\newcommand{\calM}{\mathcal{M}}
\newcommand{\calS}{\mathcal{S}}
\newcommand{\calT}{\mathcal{T}}
\newcommand{\calX}{\mathcal{X}}
\newcommand{\calY}{\mathcal{Y}}

\newcommand{\inparen}[1]{\left{ #1 \right}}
\newcommand{\probtwo}[2]{\mathsf{Pr}_{#1}\left[ #2 \right]}
\newcommand{\set}[1]{\left\{ #1 \right\}}
\newcommand{\ct}{\mathsf{ct}}
\newcommand{\twotimessadv}[1]{\mathsf{2SSAdv}\left[ #1 \right]}



\newlength{\protowidth}
\newcommand{\pprotocol}[5]{
{\begin{figure*}[#3]
\begin{center}
\setlength{\protowidth}{\textwidth}
\addtolength{\protowidth}{-3\intextsep}

\fbox{
        \small
        \hbox{\quad
        \begin{minipage}{\protowidth}
    \begin{center}
    {\bf #1}
    \end{center}
        #5
        \end{minipage}
        \quad}

        }
        
\end{center}
\vspace{-4ex}
\caption{{#4} #2}
\end{figure*}
} }

% the first arg is name of security game
% the second arg is caption
% the third arg is the game description
% the label needs to be included 
\newcommand{\securitygame}[4]{
   \pprotocol{#1}{#2}{ht!}{#3}{#4}
}

\newcommand{\constr}[4]{
   \pprotocol{#1}{#2}{tbh!}{#3}{#4}
}

\begin{document}

\noindent
\large\textbf{Anish Banerjee, Shankh Gupta} \hfill \textbf{Problem Set - 2}   \\
\normalsize COL759: Cryptography \hfill September 2023\\
\hr


\prob{1: Cryptosystems secure against side-channel attacks}
\begin{solution}
    Consider the PRF $F':\bit^{n+1}\times\bit^n\rightarrow\bit^n$
    $$F'(k||b_k,x)=
    \begin{cases}
        F(k,0^n)[1\dots n-1]||b_k \hspace{50pt} \text{if } x=0^n  \\
        F(k,x) \hspace{120pt} \text{Otherwise}
    \end{cases}$$
    In other words, the last bit of $F(k,0^n)$ has been replaced with the last bit of the key.

    \begin{enumerate}[(a)]
        \item Let $\calA$ be an adversary which breaks the PRF security of $F'$ with non-negligible advantage $\epsilon$. We will build a reduction $\calB$ which breaks the PRF security of $F$ with the same advantage.
        \securitygame{Problem 1(a)}{Reduction for Problem 1(a)}{\label{red:p1a}}
        {
            \begin{itemize}
                \item Challenger picks a uniformly random bit $b \gets \bit$ and a key $k\gets \calK$.
                \item $\calB$ samples a random $b_k \gets \bit$.
                \item The adversary $\calA$ makes polynomially many queries $\{x_i\}$ to $\calB$ who passes them to the challenger. Challenger replies as in the PRF Game.
                \item Upon receiving the response $y_i$ of each query, $\calB$ checks if $x_i=0$. If so, it modifies $y_i$ by exchanging its last bit with $b_k$. Otherwise, it just passes $y_i$ to $\calA$.
                \item After polynomially many queries, $\calB$ forwards the response send by $\calA$ $(b')$ and wins if $b=b'$.
            \end{itemize}
        }
        \begin{figure}[!ht]
            \centering
            \includegraphics[scale=0.25]{images/Reduction1.jpg}
            \caption{Image for Problem 1(a) Image}
            \label{fig:red1}
        \end{figure}

        When the challenger chooses $b=0$, the game is equivalent to the challenger choosing 0 in PRF game of $F'$. 
        \begin{center}
            Pr[$b'=0|b=0$]=Pr[$\calA$ outputs zero when the challenger chooses 0 in PRF game of $F'$]
        \end{center}
        When the challenger chooses $b=1$, $\calA$ receives the output of a random function for all $x_i\neq0^n$. For $x_i=0^n$, the output received is $r||b_k$. Since $b_k$ is choosen randomly, this too is random.
        \begin{center}
            Pr[$b'=0|b=1$]=Pr[$\calA$ outputs zero when the challenger chooses 1 in PRF game of $F'$]
        \end{center}
        Hence we can conclude,
        $$\prf[\calB,F]=\prf[\calA,F']$$

        \item We will show that $F'$ does not satisfy 1-leakage resilience by constructing an adversary $\calA'$ who makes a leakage query for the last bit of the key and breaks $F'$.
        \begin{itemize}
            \item \textbf{Leakage Query:} $\calA'$ makes a query for the last bit of the key and receives $b_k$ from the challenger.
            \item \textbf{PRF Query:} $\calA'$ queries for the $x=0^n$ and receives $y_i$. He checks if the last bit of $y_i$ is $b_k$. If yes it outputs $b'=0$ (PRF), otherwise it outputs $b'=1$ (Random Function). 
        \end{itemize}
        From the game and definition of $F'$, it is evident that:
        $$\Pr[b'=0|b=0]=1$$
        When the challenger chooses $b=0$, the evaluation of a random function at $0^n$ can have its last bit as 0 or 1 with 1/2 probability. So,
        $$\Pr[b'=0|b=1]=\frac12$$
        And the advantage of $\calA'$ is
        $$\prf[\calA,F']=\Pr[b'=0|b=0]-\Pr[b'=0|b=1]=1-\frac12=\frac12$$
        Which is non-negligible.
    \end{enumerate}
\end{solution}


\prob{2 : MACs: unique queries vs non-unique queries}
\begin{solution}
    \begin{enumerate}[(a)]
    \item We propose the following MAC scheme $I_{uq} =$ (Sign$_{uq}$, Verify$_{uq}$) that is UFCMA-Unique secure, but not UFCMA secure :
    $$sign(k, m; r) = (r, F(k_1, r) || F(k_2, m \oplus r))$$ and 
    $$verify(k, m, \sigma; r) = \begin{cases}
        1 & \text{if } \sigma = F(k_1, r) || F(k_2, m \oplus r) \\
        0 & \text{otherwise}
    \end{cases}$$
    where $k = (k_1, k_2) \in \{0, 1\}^n$, $m \in \{0, 1\}^n$ and $r \in \{0, 1\}^n$

    \item To prove the UFCMA-Unique security of $I_{uq} =$, we first build the following sequence of hybrid games : \\

    \textit{Game 0:} This is the original security game.
    \begin{itemize}
        \item \textbf{Setup Phase :} Challenger chooses a PRF key $k = (k_1, k_2)$.
        \item \textbf{Query Phase :} Adversary sends polynomially many queries. For the $i^{th}$ query $m_i$, the challenger chooses a random string $r_i \leftarrow \{0, 1\}^n$ and sends $\sigma_i = sign(k, m_i ; r_i)$ to the adversary.
        \item \textbf{Forgery :} Finally, the adversary outputs $(m^*, \sigma^*)$ such that $(m^*, \sigma^*) \neq (m_i, \sigma_i)$ for all i.
    \end{itemize}
    let the winning probability of Adversary $\calA$ in game 0 be $p_0$. \\

    \textit{Game 1:} In this game, Challenger replaces one of the PRFs with a random function.
    \begin{itemize}
        \item \textbf{Setup Phase :} Challenger chooses a PRF key $k = (k_1, k_2)$ and samples $f \leftarrow Func[\calX, \calY]$. (here $\calY = \{0, 1\}^n$ and $\calX = \{0, 1\}^n$)
        \item \textbf{Query Phase :} Adversary sends polynomially many queries. For the $i^{th}$ query $m_i$, the challenger chooses a random string $r_i \leftarrow \{0, 1\}^n$ and sends $\sigma_i = f(r_i) || F(k_2, m \oplus r_i)$ to the adversary.
        \item \textbf{Forgery :} Finally, the adversary outputs $(m^*, \sigma^*)$ such that $(m^*, \sigma^*) \neq (m_i, \sigma_i)$ for all i.
    \end{itemize}
    let the winning probability of Adversary $\calA$ in game 1 be $p_1$. \\

    \textit{Game 2:} In this game, Challenger replaces both the PRFs with random functions.
    \begin{itemize}
        \item \textbf{Setup Phase :} Challenger samples $f_1 \leftarrow Func[\calX, \calY]$ and $f_2 \leftarrow Func[\calX, \calY]$.
        \item \textbf{Query Phase :} Adversary sends polynomially many queries. For the $i^{th}$ query $m_i$, the challenger chooses a random string $r_i \leftarrow \{0, 1\}^n$ and sends $\sigma_i = f(r_i) || f(m \oplus r_i)$ to the adversary.
        \item \textbf{Forgery :} Finally, the adversary outputs $(m^*, \sigma^*)$ such that $(m^*, \sigma^*) \neq (m_i, \sigma_i)$ for all i.
    \end{itemize}
    let the winning probability of Adversary $\calA$ in game 2 be $p_2$. \\
    
    \textit{Claim 1 :} $|p_0 - p_1|$ is negligible.\\
    \textbf{Proof:} We will show a reduction algorithm $\calB$ which uses $\calA$ and has PRFAdv = $|p_0 - p_1|$.\\
    The reduction algorithm receives signing queries ($m_i$) from $\calA$. It samples $r_i$ and a key $k$ and forwards it to the PRF challenger. It the receives $f(r_i)$ from the PRF challenger and calculates $\sigma_i = f(r_i) || F(k, m \oplus r_i)$. It sends the signature $\sigma_i$ to the adversary. The adversary then guesses a bit b, where b = 0 for Game 0 and b = 1 for Game 1. The reduction $\calB$ then forwards the bit b to the PRF challenger.\\
    If the PRF challenger chooses pseudorandom, this corresponds to Game 0 for the adversary $\calA$ and it outputs 0 with a probability $p_0$. On the other hand, if PRF challenger chooses completely random, then it corresponds to Game 1 for the adversary and it outputs 1 with probability $p_1$. So $$PRFAdv[B, F] = |p_0 - p_1|$$.
    Hence, $|p_0 - p_1|$ is negligible as it is given that F is a secure PRF.\\

    \textit{Claim 2:} $|p_1 - p_2|$ is negligible.\\
    \textbf{Proof:} We will show a reduction algorithm $\calB$ which uses $\calA$ and has PRFAdv = $|p_1 - p_2|$.\\
    The reduction algorithm receives signing queries ($m_i$) from $\calA$. It samples $r_i$ and a random function $f' \leftarrow Func[\calX, \calY]$. It then forwards $m_i \oplus r_i$ to the PRF challenger. It then receives $f(r_i \oplus m_i)$ from the PRF challenger and calculates $\sigma_i = f'(r_i) || f(m \oplus r_i)$. It sends the signature $\sigma_i$ to the adversary. The adversary then guesses a bit b, where b = 0 for Game 1 and b = 1 for Game 2. The reduction $\calB$ then forwards the bit b to the PRF challenger.\\
    If the PRF challenger chooses pseudorandom, this corresponds to Game 1 for the adversary $\calA$ and it outputs 0 with a probability $p_1$. On the other hand, if PRF challenger chooses completely random, then it corresponds to Game 2 for the adversary and it outputs 1 with probability $p_2$. So $$PRFAdv[B, F] = |p_1 - p_2|$$.
    Hence, $|p_1 - p_2|$ is negligible as it is given that F is a secure PRF.\\

    \textit{Claim 3:} For any Adversary $\calA$, $p_2 = \frac{1}{|\calY|^2}$. (note that, $|\calY| = 2^n$).\\
    \textbf{Proof :} As, $f_1$ and $f_2$ are completely random functions, $f(x)$ for any x is equivalent to choosing a random output $y \in \calY$.\\
    In order to win the game, the output by the adversary $(m^*, \sigma^*)$ must satisfy $\sigma^*_1 = f_1(r)$ and $\sigma^*_2 = f_2(m_i \oplus r)$ ($\sigma^*_1$ and $\sigma^*_2$ are the first n and the last n bits of $\sigma^*$ respectively). Since $f_1$ and $f_2$ are completely random, the probability of this happening would be $\frac{1}{|\calY|^2}$.\\

    Thus, by all the above claims, we can prove that $I_{uq}$ is a UFCMA-Unique MAC scheme.

    \item We will show this using a counter example:\\
    \begin{itemize}
        \item The adversary queries for m twice and receives $\sigma_1 = F(k_1, r) || F(k_2, m \oplus r)$ adn $\sigma_2 = F(k_1, r') || F(k_2, m \oplus r')$.
        \item It then sends $(m \oplus r \oplus r', (r', \sigma_{21} || \sigma_{12}))$. (where $\sigma_{21}$ are the first n bits of $\sigma_{2}$ and similarly $\sigma_{12}$ are the last n bits of $\sigma_{1}$.)
    \end{itemize}

    Since $(m \oplus r \oplus r') \oplus r' = (m \oplus r)$, the above forgery is valid. The only case where this forgery would not work is when both $r$ and $r'$ are same. This happens with a very small probability of $\frac{1}{2^{2n}}$ and hence is negligible. So the adversary wins with a \textbf{probability of almost 1}. \\
    Also the adversary can continue sending m as signing queries until it receives two such signatures with distinct randomness. Then the adversary will win with probability exactly equal to 1.
 \end{enumerate}
\end{solution}


\prob{3 : A mistake in the lecture notes}
\begin{solution}
    According to the given flawed argument, for any (even unbounded) adversary $\calA$ who wins the MAC game with verification queries ($\macvq$) with advantage $\epsilon$, we can construct an adversary $\calB$ who wins the MAC game without verification queries ($\mac$) with probability $\epsilon$. However, we will show an adversary $\calA'$ who wins  $macvq$ with advantage 1 but the reduction $\calB$ cannot use it to win $\mac$.

    The key observation here is that since every message has a unique signature, $\calB$ cannot send a forgery of a message which it has already queried. 
    \begin{itemize}
        \item $\calA'$ sends verification queries $(\verify , m, \sigma) \forall \sigma\in\calT$ where $\calT$ is the signature space.
        \item For the first verification query, $\calB$ queries the challenger to obtain the signature $\sigma^\star$, and checks all the verification queries against this.
    \end{itemize} 
    One of the queries by $\calA'$ must be $(\verify , m, \sigma^\star)$ and thus he wins the $\macvq$ game. However, $\calB$ cannot use this forgery to win the $\mac$ game since he has already queried it from the challenger.

\end{solution}


\prob{4 : Even-Mansour instantiated with a bad permutation}
\begin{solution}
   The key observation here is that for any query $x_i$ which results in an output $y_i$:
   $$(y_i-k_2)(x_i+k_1)=1 \mod p$$
   So, we query the oracle at 3 points 0,1,2 and form three equations:
   $$(y_0-k_2)(0+k_1)=1 \mod p$$
   $$(y_1-k_2)(1+k_1)=1 \mod p$$
   $$(y_2-k_2)(2+k_1)=1 \mod p$$
   On solving we can calculate $(k_1,k_2)$
   $$k_1=2(y_1-y_2)(y_0+y_2-2y_1)^{-1}$$
   $$k_2=y_1-k_1(y_0-y_1)$$
   Now, we can just query $\pi$ and check if these $(k1,k2)$ satisfy $P(x_i)=\pi(x_i+k_1)+k_2$
   
   \noindent\textbf{Note:} all the additions and multiplications are modulo p
\end{solution}


\prob{5 : 3-round Luby-Rackoff with inversion queries}
\begin{solution}
   To launch an attack against the 3-round Luby-Rackoff permutation, we will need 3 oracle queries.
   \begin{itemize}
       \item permute(x, y)
       \item permute(x, z)
       \item inverse\_permute($\alpha$, $\beta$)
   \end{itemize}
   where x, y and z are some 16 bytes array and $y \neq z$. Let permute(x, y) = $(\gamma_1, \delta_1)$ and permute(x, z) = $(\gamma_2, \delta_2)$. \\
   We will choose $\alpha = \gamma_2$ and $\beta = \delta_2 \oplus y \oplus z$. Let inverse\_permute($\alpha$, $\beta$) = ($m$, $n$).//

   If $ m = x \oplus \gamma_1 \oplus \gamma_2$ then the adversary outputs 0 (indicating PRP) else the adversary attack outputs 1 (indicating Completely random).\\

   This attack works because for 3-round Luby-Rackoff $$PRP(x, y) = (x \oplus F(k_2, y \oplus F(k_1, x)), y \oplus F(k_1, x) \oplus F(k_3, x \oplus F(k_2, y \oplus F(k_1, x))$$
   and 
   $$PRP^{-1}(u, v) = (u \oplus F(k_2, v \oplus F(k_3, u), ... )$$

   on substituting $(u, v) = (\alpha, \beta)$ we get $m = x \oplus F(k_2, y \oplus F(k_1, x)) \oplus F(k_2, z \oplus F(k_1, x))$ which is same as $x \oplus \gamma_1 \oplus \gamma_2$.
\end{solution}


\prob{6 : CBC mode with bad initialization}
\begin{solution}
   Suppose the given ciphertext is $(ct_0, ct_1, ct_2)$. Then 
   $$ct_0=\AES(k,k\oplus m_0)$$
   $$ct_1=\AES(k,ct_0\oplus m_1)$$
   $$ct_2=\AES(k,ct_1\oplus m_2)$$
   
   The attacker passes $(ct_0, ct_0, ct_0)$ to the decrypt query and recieves $(m'_0,m'_1,m'_2)$. Note that $m'_0=m_0$ He can simply recover the key by 
   $$m'_1=k\oplus m'_0\oplus ct_0$$
   $$\implies k=m'_1\oplus m'_0\oplus ct_0$$
\end{solution}


\prob{Part B : Coding Problem}


\end{document}
