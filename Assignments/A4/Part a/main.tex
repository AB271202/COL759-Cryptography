\documentclass[a4paper, 11pt]{article}
\usepackage{mdframed}
\usepackage{comment} 
\usepackage{xparse}
\usepackage{tcolorbox}
\usepackage{lipsum} %This package just generates Lorem Ipsum filler text. 
\usepackage{fullpage} % changes the margin
\usepackage[a4paper, total={7in, 10in}]{geometry}
\usepackage[fleqn]{amsmath}
\usepackage{amssymb,amsthm}  % assumes amsmath package installed
\newtheorem{theorem}{Theorem}
\newtheorem{corollary}{Corollary}
\usepackage{graphicx}
\usepackage{tikz}
\usetikzlibrary{arrows}
\usepackage{verbatim}
\usepackage[numbered]{mcode}
\usepackage{float}
\usepackage{tikz}
    \usetikzlibrary{shapes,arrows}
    \usetikzlibrary{arrows,calc,positioning}

    \tikzset{
        block/.style = {draw, rectangle,
            minimum height=1cm,
            minimum width=1.5cm},
        input/.style = {coordinate,node distance=1cm},
        output/.style = {coordinate,node distance=4cm},
        arrow/.style={draw, -latex,node distance=2cm},
        pinstyle/.style = {pin edge={latex-, black,node distance=2cm}},
        sum/.style = {draw, circle, node distance=1cm},
    }
\usepackage{xcolor}
\usepackage{mdframed}
\usepackage[shortlabels]{enumitem}
\usepackage{indentfirst}
\usepackage{hyperref}
\usepackage[capitalize, nameinlink]{cleveref}
\renewcommand{\thesubsection}{\thesection.\alph{subsection}}

\newenvironment{problem}[2][Problem]
    { \begin{mdframed}[backgroundcolor=gray!20] \textbf{#1 #2} \\}
    {  \end{mdframed}}

\newenvironment{reduction}
    { \begin{mdframed}[backgroundcolor=blue!20] \\}
    {  \end{mdframed}}
% Define solution environment

\newcommand{\hr}{\noindent\rule{7in}{2.8pt}}
\newenvironment{solution}
    {\textit{Solution:}}
    {\clearpage}
\newcommand{\prob}[1]{\begin{mdframed}[backgroundcolor=gray!20] \textbf{Problem #1}\end{mdframed}}
\renewcommand{\qed}{\quad\qedsymbol}
\newcommand{\bit}{\left\{0, 1\right\}}
\newcommand{\ct}{\mathsf{ct}}
\newcommand{\hyb}{\mathsf{Hyb}}
\newcommand{\enc}{\mathsf{Enc}}
\newcommand{\enct}{\mathsf{Enc-two}}
\newcommand{\dec}{\mathsf{Dec}}
\newcommand{\dect}{\mathsf{Dec-two}}
\newcommand{\negl}{\mathsf{negl}}
\newcommand{\prf}{\mathsf{PRFAdv}}
\newcommand{\prg}{\mathsf{PRGAdv}}
\newcommand{\poly}{\mathsf{poly}}
\newcommand{\sign}{\mathsf{Sign}}
\newcommand{\verif}{\mathsf{Verify}}
\newcommand{\ord}{\mathsf{ord}}
\newcommand{\ddh}{\mathsf{DDH}}
\newcommand{\N}{\mathbb{N}}
\newcommand{\R}{\mathbb{R}}
\newcommand{\Z}{\mathbb{Z}}
\newcommand{\T}{\mathbb{T}}
\newcommand{\G}{\mathbb{G}}

\newcommand{\calA}{\mathcal{A}}
\newcommand{\calB}{\mathcal{B}}
\newcommand{\calC}{\mathcal{C}}
\newcommand{\calD}{\mathcal{D}}
\newcommand{\calE}{\mathcal{E}}
\newcommand{\calF}{\mathcal{F}}
\newcommand{\calG}{\mathcal{G}}
\newcommand{\calH}{\mathcal{H}}
\newcommand{\calI}{\mathcal{I}}
\newcommand{\calJ}{\mathcal{J}}
\newcommand{\calK}{\mathcal{K}}
\newcommand{\calM}{\mathcal{M}}
\newcommand{\calS}{\mathcal{S}}
\newcommand{\calX}{\mathcal{X}}
\newcommand{\calY}{\mathcal{Y}}
\newcommand{\calR}{\mathcal{R}}
\newcommand{\calT}{\mathcal{T}}

\newcommand{\inparen}[1]{\left{ #1 \right}}
\newcommand{\probtwo}[2]{\mathsf{Pr}_{#1}\left[ #2 \right]}
\newcommand{\set}[1]{\left\{ #1 \right\}}
\newcommand{\twotimessadv}[1]{\mathsf{2SSAdv}\left[ #1 \right]}

\NewDocumentEnvironment{world}{ o }{%
  \begin{mdframed}[
    backgroundcolor=blue!10,
    innertopmargin=15pt,
    innerbottommargin=15pt,
    innerleftmargin=15pt,
    innerrightmargin=15pt,
  ]%
  \IfNoValueTF{#1}{%
    % If no title is provided
    \centering
  }{%
    % If a title is provided
    \centering\textbf{#1}\par
  }%
}{%
  \end{mdframed}%
}

\newlength{\protowidth}
\newcommand{\pprotocol}[5]{
{\begin{figure*}[#3]
\begin{center}
\setlength{\protowidth}{\textwidth}
\addtolength{\protowidth}{-3\intextsep}

\fbox{
        \small
        \hbox{\quad
        \begin{minipage}{\protowidth}
    \begin{center}
    {\bf #1}
    \end{center}
        #5
        \end{minipage}
        \quad}

        }
        
\end{center}
\vspace{-4ex}
\caption{{#4} #2}
\end{figure*}
} }

% the first arg is name of security game
% the second arg is caption
% the third arg is the game description
% the label needs to be included 
\newcommand{\securitygame}[4]{
   \pprotocol{#1}{#2}{ht!}{#3}{#4}
}

\newcommand{\constr}[4]{
   \pprotocol{#1}{#2}{tbh!}{#3}{#4}
}

\begin{document}

\noindent
\large\textbf{Anish Banerjee, Shankh Gupta} \hfill \textbf{Problem Set - 4}   \\
\normalsize COL759: Cryptography \hfill November 2023\\
\hr


\prob{: Collision Resistant Hashing based on number-theoretic assumptions}
\begin{solution}
    \begin{enumerate}[(a)]
        \item Let $\calA$ be an adversary that can break the collision resistance property of the Hash function $H:\G^\lambda\times\Z_q^\lambda\rightarrow\G$ 
        $$H((g_1, g_2, \dots, g_\lambda), (\alpha_1, \alpha_2,\dots,\alpha_\lambda))=\prod_{i=1}^\lambda g_i^{\alpha_i}$$
        with non-negligible probability $\epsilon$. We will use $\calA$ to build an adversary $\calB$ which breaks DLOG assumption. Consider the reduction:
            \securitygame{Reduction for (a)}{Reduction for (a)}{\label{red:a}}
    {        \begin{itemize}
            \item The RSA Challenger sends the tuple $(q, g, g^\alpha)$ to $\calB$
            \item $\calB$ samples $i^*\gets[\lambda]$ and $\beta_1, \beta_2, \dots, \beta_{\lambda-1}\gets\Z_q$.
            \item $\calB$ sends the key $(g^{\beta_1}, g^{\beta_2}\dots g^\alpha, \dots g^{\beta_{\lambda-1}})$ where $g^\alpha$ is the $i^*$th element of the tuple to $\calA$
            \item $\calA$ responds with a collision $(x_1, x_2, \dots,x_{i^*}, \dots, x_{\lambda-1}), (y_1, y_2, \dots,y_{i^*},\dots y_{\lambda-1})$ such that
            \begin{equation}
            \left(\prod_{i=1}^{\lambda-1} g^{\beta_i x_i}\right)g^{\alpha x_{i^*}}=\left(\prod_{i=1}^{\lambda-1} g^{\beta_i y_i}\right)g^{\alpha y_{i^*}}
            \end{equation}
            \item Using the collision, $\calB$ computes $\alpha$ as follows:
            
            Since $g^m=g^n\implies m=n$ for the generator $g$, we have 

            \begin{equation}  
            \left(\sum_{i=1}^{\lambda-1} {\beta_i x_i}\right)+{\alpha x_{i^*}}=\left(\sum_{i=1}^{\lambda-1} {\beta_i y_i}\right)+{\alpha y_{i^*}}
            \end{equation}
        
            \begin{equation}
            \alpha=(y_{i^*}-x_{i^*})^{-1}\sum_{i=1}^{\lambda-1} {\beta_i (x_i-y_i)}
            \end{equation}
            Assuming $y_{i^*}\neq x_{i^*}$ (explained below). Note that all the operations in eqn (2) and (3) are in $\Z_q$
        \end{itemize}
    }

        As mentioned above, the reduction works only when $y_{i^*}\neq x_{i^*}$. 
        Observe that at least two elements of the tuples $(x_1, x_2, \dots,x_{i^*}, \dots, x_{\lambda-1}), (y_1, y_2, \dots,y_{i^*},\dots y_{\lambda-1})$ must be distinct (if only one element is different, then equation (2) makes them equal). 
        Since the index $i^*$ is chosen randomly, probability that $i^*$ matches with the distinct elements is $$\Pr[\text{Match}]\geq\frac{2}{\lambda}$$
        Hence the winning probability of the reduction is 
        $\Pr[\text{Win}]\geq\frac{2\epsilon}{\lambda}$

\textbf{NOTES:}
\begin{itemize}
    \item Since $g$ is the generator of the group, any random element can be written in the form $g^\beta$ where $\beta$ is sampled randomly. So, the key received by $\calA$ is randomly generated.
    \item It is essential to sample $i^*$ randomly. If we chose it to be fixed, then there may exist $\calA$ which always outputs collision tuples matching at that specific index.
\end{itemize}



% -------------------------------------------------------
        
        \item Let us define an efficient adversary $\calA$ which can break the CRHF security of $h_{N, e, z}$. In other words, given input $(N, e, z)$, adversary $\calA$ can output a collision $(x_1, y_1)$ and $(x_2, y_2)$, with $(x_1, y_1) \neq (x_2, y_2)$, such that $h_{N, e, z}(x_1, y_1) = h_{N, e, z}(x_2, y_2)$. We will show that if there exists such an adversary $\calA$, then it is possible to construct a reduction algorithm $\calB$ which uses $\calA$ to break the RSA problem. \\
        The algorithm $\calB$ interacts with $\calA$ and a challenger $\calC$ as follows :
                    \securitygame{Reduction for (b)}{Reduction for (b)}{\label{red:b}}
    {
        \begin{itemize}
            \item The challenger $\calC$ runs the setup to obtain the key $N$, $e$ and a random integer $z \gets \Z^*_N$. It forwards $(N, e, z)$ to reduction $\calB$ which in turn forwards this to the adversary $\calA$.
            \item The adversary then sends its collision as $(x_1, y_1)$ and $(x_2, y_2)$ to the reduction $\calB$.
            \item $\calB$ uses the output of adversary $\calA$ to compute the $e^{th}$ root of $z$, which it then forwards to the challenger and breaks the RSA assumption.
        \end{itemize}
}

        $\calB$, given a collision, computes the $e^{th}$ root of $z$ as follows:
        \begin{itemize}
            \item $\calA$ forwards its collision $(x_1, y_1)$ and $(x_2, y_2)$ to $\calB$ such that $$x_1^e \cdot z^{y_1} = x_2^e \cdot z^{y_2}$$
            Observe that if $y_1=y_2$ then $x_1^e=x_2^e$. This would result in $x_1$ being equal to $x_2$, since $e$ is co-prime to $\phi(N)$ and the $e^{th}$ root (i.e. the inverse of $e$) should be unique. Therefore it is not possible for adversary to output a collision of the form $(x_1, y), (x_2, y), x_1\neq x_2$. Hence we can safely assume $y_2>y_1$ without loss of generality.
            \item The reduction then computes the inverse of $x_2$ and $z$. Since we have $x_1^e \cdot z^{y_1} = x_2^e \cdot z^{y_2}$, we can multiply both sides by $(x_2^{-1})^e$ and $(z^{-1})^{y_1}$ to get $$(x_1x_2^{-1})^e = (z)^{y_2 - y_1}$$ 
            Let $a=x_1x_2^{-1}$ and $b=y_2 - y_1$. So the reduction has computed $a \in \Z^*_N$ and $b \in \Z^*_e$, such that $$a^e = z^b$$.
            \item $\calB$ now needs to compute the $b^{th}$ root of $a$. Observe that since $b$ and $e$ are co-prime, by the Extended Euclid's algorithm we can find $m$ and $n$ such that:
            $$mb+ne=1$$
            Now,
            $$z^1=z^{mb+ne}=a^{me}z^{ne}=(a^mz^n)^e$$
            Thus the $e^{th}$ root of z is $a^mz^n$.
        \end{itemize}
        \textbf{NOTES:} 
        \begin{itemize}
            \item Since $x_2$ and $z$ both belong to $\Z^*_N$, so we can compute using Extended Euclid's Algorithm $s$, $t$, $u$ and $v$ such that $$sx_2 + tN = 1$$ and $$uz + vN = 1$$ Using these we can efficiently compute the inverse of both $x_2$ and $z$.

            \item $b \in \Z^*_e$. This is because both $y_1$ and $y_2$ belongs to $\Z_e$ and assuming $y_2 > y_1$  we can say that $b < e$. Also, it is not possible for $y_1$ to be equal to $y_2$ (as explained above). Hence $0<b < e$, which implies it belongs to $\Z^*_e$.
        \end{itemize}
    
    \end{enumerate}
\end{solution}

\end{document}
