\documentclass[a4paper, 11pt]{article}
\usepackage{comment} 
\usepackage{lipsum} %This package just generates Lorem Ipsum filler text. 
\usepackage{fullpage} % changes the margin
\usepackage[a4paper, total={7in, 10in}]{geometry}
\usepackage[fleqn]{amsmath}
\usepackage{amssymb,amsthm}  % assumes amsmath package installed
\newtheorem{theorem}{Theorem}
\newtheorem{corollary}{Corollary}
\usepackage{graphicx}
\usepackage{tikz}
\usetikzlibrary{arrows}
\usepackage{verbatim}
\usepackage[numbered]{mcode}
\usepackage{float}
\usepackage{tikz}
    \usetikzlibrary{shapes,arrows}
    \usetikzlibrary{arrows,calc,positioning}

    \tikzset{
        block/.style = {draw, rectangle,
            minimum height=1cm,
            minimum width=1.5cm},
        input/.style = {coordinate,node distance=1cm},
        output/.style = {coordinate,node distance=4cm},
        arrow/.style={draw, -latex,node distance=2cm},
        pinstyle/.style = {pin edge={latex-, black,node distance=2cm}},
        sum/.style = {draw, circle, node distance=1cm},
    }
\usepackage{xcolor}
\usepackage{mdframed}
\usepackage[shortlabels]{enumitem}
\usepackage{indentfirst}
\usepackage{hyperref}
    
\renewcommand{\thesubsection}{\thesection.\alph{subsection}}

\newenvironment{problem}[2][Problem]
    { \begin{mdframed}[backgroundcolor=gray!20] \textbf{#1 #2} \\}
    {  \end{mdframed}}

% Define solution environment
\newenvironment{solution}
    {\textit{Solution:}}
    {}
\newcommand{\hr}{\noindent\rule{7in}{2.8pt}}
\renewcommand{\qed}{\quad\qedsymbol}
%%%%%%%%%%%%%%%%%%%%%%%%%%%%%%%%%%%%%%%%%%%%%%%%%%%%%%%%%%%%%%%%%%%%%%%%%%%%%%%%%%%%%%%%%%%%%%%%%%%%%%%%%%%%%%%%%%%%%%%%%%%%%%%%%%%%%%%%
\begin{document}
%Header-Make sure you update this information!!!!
\noindent
%%%%%%%%%%%%%%%%%%%%%%%%%%%%%%%%%%%%%%%%%%%%%%%%%%%%%%%%%%%%%%%%%%%%%%%%%%%%%%%%%%%%%%%%%%%%%%%%%%%%%%%%%%%%%%%%%%%%%%%%%%%%%%%%%%%%%%%%
\large\textbf{Anish Banerjee, Shankh Gupta} \hfill \textbf{Problem Set - 1}   \\
\normalsize COL759: Cryptography \hfill August 2023\\
\hr
%%%%%%%%%%%%%%%%%%%%%%%%%%%%%%%%%%%%%%%%%%%%%%%%%%%%%%%%%%%%%%%%%%%%%%%%%%%%%%%%%%%%%%%%%%%%%%%%%%%%%%%%%%%%%%%%%%%%%%%%%%%%%%%%%%%%%%%%
% Problem 1
%%%%%%%%%%%%%%%%%%%%%%%%%%%%%%%%%%%%%%%%%%%%%%%%%%%%%%%%%%%%%%%%%%%%%%%%%%%%%%%%%%%%%%%%%%%%%%%%%%%%%%%%%%%%%%%%%%%%%%%%%%%%%%%%%%%%%%%%
\begin{problem}{1}
An IPL tournament is played between $n$ cricket teams, where each team plays exactly one match with every other team. How many matches are played? (This is easy.) Assume that no match ends in a tie. We say that a subset $S$ of teams is \textit{consistent} if it is possible to order teams in $S$ as $T_1,\ldots,T_{|S|}$ (think of this as the strongest to weakest ordering) such that for every $i,j$ with $1\leq i<j\leq|S|$, $T_i$ beats $T_j$. Prove that irrespective of the outcomes of the matches, there always exists a consistent subset $S$ with $|S|\geq\log_2 n$.
\end{problem}
\begin{solution}
The number of matches played is given by ${n \choose 2} =  \frac{n(n-1)}{2}$. For the second claim, we proceed using the Strong Induction. Consider the claim: 
\begin{align*}
    p(k): \text{In a tournament of } k \text{ teams, always exists a consistent subset } S \text{ such that } |S| \geq \log_2 (n+1)
\end{align*}
\underline{Base Case:} For $n=2$, there are only two teams $T_1$ and $T_2$. Say $T_1$ wins against $T_2$. Then $\{ T_1, T_2 \}$ form a consistent subset with $T_1T_2$ being the required ordering. $|S| = 2 \geq \log_2 3$. So $P(2)$ is true. For $n = 3$, consider the outcome of any one match. Say $T_1$ wins against $T_2$. Consider the set $S = \{ T_1, T_2 \} $, which is consistent under the ordering $T_1 T_2$. Also, $|S| = 2 \geq \log_2{(n + 1)} = \log_2{4}$.
\\
\\
\underline{Induction Hypothesis:} Let $P(k)$ be true $\forall \ k < n$, $k \geq 2$.
\\
\\
\underline{Induction Step:} Consider a tournament of $n$ teams. We will construct a consistent subset for this tournament. Note that there will be $\frac{n(n-1)}{2}$ wins distributed amongst $n$ teams, and so by Pigeon-Hole principle, exists a team $T$ such that $T$ wins at least $\lceil \frac{n(n-1)}{2n} \rceil = \lceil \frac{n-1}{2} \rceil$ matches. 
\\
\\
Suppose the set of teams $T$ wins against is $X$. Since $\lceil \frac{n-1}{2} \rceil < |X| < n$, by the induction hypothesis, there exists a consistent subset $S$ of the teams in $X$ such that, 
$$|S| \geq \log_2{(|X|+1)}$$
Thus, $\exists$ ordering $T_1 T_2 ... T_{|S|} $ such that $T_i$ beats $T_j \ \forall \ i < j $. Note that $T$ wins against $T_k \ \forall \ T_k \in S$ and so the set $S' = \{T \} \cup X$ is also consistent, with the required ordering $TT_1 T_2 ... T_{|S|}$. Now we have, $|S'| = |S|+1 \geq \log_2{(|X|+1)} + 1$. 
Since $\lceil \frac{n-1}{2} \rceil < |X|$ we have, 
$$|S'| \geq \log_2{\left(\left \lceil \frac{n-1}{2} \right \rceil+1 \right)} + 1 \geq \log_2{\left(\frac{n-1}{2} + 1 \right)} + 1 \geq \log_2{(n+1)}$$
Thus, we have $S' = T \cup S$ is the required consistent subset. $\blacksquare$
\end{solution} 
\\
\hr

\begin{problem}{2 : Secure/Insecure PRGs PRFs}\end{problem}
\begin{solution}

\end{solution} 

\hr

\begin{problem}{3}
Prove ``Claim 2'' from the proof of Schr\"{o}der-Bernstein Theorem discussed in Lecture 5. Here is the statement of the claim. Let $A$ and $B$ be infinite sets, $f$ be an injection from $A$ to $B$, and $g$ be an injection from $B$ to $A$. Let $B'=\{b\in B\mid\exists b^*\in B\setminus\text{Im}(f)\text{ }\exists k\in\mathbb{N}\cup\{0\}:\text{ }b=(f\circ g)^k(b^*)\}$, and $A'=\{g(b)\mid b\in B'\}$. Then for every $b\in B$, the following statements are equivalent.
\begin{enumerate}
\item $b\in B'$.
\item If $f^{-1}(b)$ exists, then it is in $A'$.
\item $g(b)\in A'$.
\end{enumerate}
\end{problem}
\begin{solution} \textbf{($1 \implies 2$)} Consider any $b \in B'$. By definition, suppose $a = f^{-1}(b)$, or $f(a) = b$. Now by the definition of $B'$, we have that $b= (f\circ g)^k(b^*)\ $ for some $k \in \mathbb{N}\cup \{0\}$ and $b^* \in B\setminus\text{Im}(f)$. Thus we may write, 
$$f(a) = (f\circ g)^k(b^*) $$
By definition of $B'$, $b \neq b^{*}$  since $b \in Im(f)$ and so $k \geq 1$. Also, by injectivity of $f$ we must have, 
$$a = g \left((f\circ g)^{k-1}(b^*)\right) = g(\bar{b}) , \  \bar{b} \in B'$$
Hence, $a \in A'$.
\\ 
\\
\textbf{($2 \implies 3$)} If $f^{-1}(b)$ exists and in $A'$, then we have $a=f^{-1}(b) = g(\bar{b})$ for some $\bar{b} \in B'$. Write $\bar{b}= (f\circ g)^k(b^*)\ $ for some $k \in \mathbb{N}\cup \{0\}$ and $b^* \in B\setminus\text{Im}(f)$, and so we have,
$$b = f(a) = f\circ g\left((f\circ g)^k(b^*)\right) = (f\circ g)^{k+1}(b^*)$$
Hence, $b \in B'$ and $g(b) \in A'$.
\\ 
\\
\textbf{($3 \implies 1$)} Since $g(b) \in A'$, by definition of $A'$ it follows that $b \in B' \ \blacksquare$
\end{solution} 
\\
\hr

\begin{problem}{4}
Given a set $A$, the set of finite length strings over $A$ is denoted by $A^*$. Prove that if $A$ is a finite set, then $A^*$ is necessarily countable. What can you say about the cardinality of $A^*$ if $A$ is countably infinite instead?
\end{problem}
\begin{solution} (a) Let $s_P$ denoted a string over a set $P$, let $s_i$ represent the i-th character of the string, and $l_s$ denote the length of a string $s$. Define the set $A_i$ as follows:
$$A_i = \left \{s_A \mid l_s = i \right \} $$ 
Note that, $$A^* = \bigcup\limits_{i=1}^{\infty} A_{i} $$
Given that $A$ is a finite set, let us suppose it's cardinality is $n$. Then, cardinality of each $A_i$ is, $|A_i| = n^i$. Thus, each $A_i$ is finite.
\\
\\
We recall that if we have a countably infinite collection of sets, each of which is countable, then $\bigcup\limits_{i=1}^{\infty} A_{i}$ is countable. Therefore, $A^*$ is countable. $\blacksquare$
\\
\\
(b) By virtue of the countability of $A$ we impose an ordering $a_1, a_2, ...$ on its elements. Define the set, 
$$B_i = \left \{ s_A \mid s_j \in \{ a_1 a_2 ... a_i \} \ \forall \ j < l_s \right \} $$ 
$B_i$ is the set of finite length strings over the finite set $\{ a_1 a_2 ... a_i \}$. By part (a), $B_i$ is countable. And we have, 
$$A^* = \bigcup\limits_{i=1}^{\infty} B_{i} $$
Once again, we have a countably infinite collection of sets, each of which is countable, and thus $\bigcup\limits_{i=1}^{\infty} B_{i}$ is countable. Therefore, $A^*$ is countable. $\blacksquare$
\end{solution} 
\\
\hr

\begin{problem}{5}
Prove by mathematical induction that every graph has at least two vertices having equal degree.
\end{problem}
\begin{solution} We proceed by induction on the number of vertices $n$ of the graph. Consider the claim: 
$$
    p(k): \text{A graph with } k \text{ vertices at least two vertices with the same degree.}
$$
\underline{Base Case:} $n=2$. If the two vertices of the graph are connected by an edge, then both have degree 1, else both have degree 0. So $P(2)$ is true.
\\
\\
\underline{Induction Hypothesis:} Let $P(n-1)$ be true for some $n \geq 3, \ n \in \mathbb{N}$.
\\
\\
\underline{Induction Step:} Consider a graph with $n$ vertices. First consider the subgraph $G'$ including any $n-1$ vertices, say $V = \{ v_1,v_2,...v_{n-1} \}$. By the induction hypothesis, $\exists$ vertices $\ v_i, v_j \in V$ of equal degree. Now when we consider the connection of the n-th vertex $v_n$ there are three cases possible: 
\paragraph{Case 1:} $v_n$ is connected to neither $v_i$ nor $v_j$. Then their degrees remains unchanged and hence equal.
\paragraph{Case 2:} $v_n$ is connected to both $v_i$ and $v_j$. Then their degrees are both increased by 1 and hence are still equal.
\paragraph{Case 3:} If $v_n$ is connected to $v_i$ but not $v_j$ (say). Let the degree of $v_k$ be represented by $d_k$. Note that $d_k \in \{ 0, 1, 2, ... n-1 \} \ \forall \ k \leq n$. However, if there exists $v: d_v = 0$ then cannot exist a vertex of degree $n-1$. Similarly if there exists $v: d_v = n-1$ then cannot exist a vertex of degree $0$. So, $d_k \in \{ 0, 1, 2, ... n-2 \}$ or $d_k \in \{ 1, 2, ... n-1 \}$. So we have here $n-1$ pigeons (possible degrees) and $n$ pigeonholes (vertices). Hence, by PHP, there must exist two vertices which have the same degree. $\blacksquare$
\end{solution} 
\\
\hr
\end{document}
 