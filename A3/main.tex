\documentclass[a4paper, 11pt]{article}
\usepackage{comment} 
\usepackage{lipsum} %This package just generates Lorem Ipsum filler text. 
\usepackage{fullpage} % changes the margin
\usepackage[a4paper, total={7in, 10in}]{geometry}
\usepackage[fleqn]{amsmath}
\usepackage{amssymb,amsthm}  % assumes amsmath package installed
\newtheorem{theorem}{Theorem}
\newtheorem{corollary}{Corollary}
\usepackage{graphicx}
\usepackage{tikz}
\usetikzlibrary{arrows}
\usepackage{verbatim}
\usepackage[numbered]{mcode}
\usepackage{float}
\usepackage{tikz}
    \usetikzlibrary{shapes,arrows}
    \usetikzlibrary{arrows,calc,positioning}

    \tikzset{
        block/.style = {draw, rectangle,
            minimum height=1cm,
            minimum width=1.5cm},
        input/.style = {coordinate,node distance=1cm},
        output/.style = {coordinate,node distance=4cm},
        arrow/.style={draw, -latex,node distance=2cm},
        pinstyle/.style = {pin edge={latex-, black,node distance=2cm}},
        sum/.style = {draw, circle, node distance=1cm},
    }
\usepackage{xcolor}
\usepackage{mdframed}
\usepackage[shortlabels]{enumitem}
\usepackage{indentfirst}
\usepackage{hyperref}
\usepackage[capitalize, nameinlink]{cleveref}
\renewcommand{\thesubsection}{\thesection.\alph{subsection}}

\newenvironment{problem}[2][Problem]
    { \begin{mdframed}[backgroundcolor=gray!20] \textbf{#1 #2} \\}
    {  \end{mdframed}}

\newenvironment{reduction}
    { \begin{mdframed}[backgroundcolor=blue!20] \\}
    {  \end{mdframed}}
% Define solution environment

\newcommand{\hr}{\noindent\rule{7in}{2.8pt}}
\newenvironment{solution}
    {\textit{Solution:}}
    {\clearpage}
\newcommand{\prob}[1]{\begin{mdframed}[backgroundcolor=gray!20] \textbf{Problem #1}\end{mdframed}}
\renewcommand{\qed}{\quad\qedsymbol}
\newcommand{\bit}{\left\{0, 1\right\}}
\newcommand{\enc}{\mathsf{Enc}}
\newcommand{\dec}{\mathsf{Dec}}
\newcommand{\negl}{\mathsf{negl}}
\newcommand{\prf}{\mathsf{PRFAdv}}
\newcommand{\prg}{\mathsf{PRGAdv}}
\newcommand{\poly}{\mathsf{poly}}
\newcommand{\ord}{\mathsf{ord}}
\newcommand{\N}{\mathbb{N}}
\newcommand{\R}{\mathbb{R}}
\newcommand{\Z}{\mathbb{Z}}

\newcommand{\calA}{\mathcal{A}}
\newcommand{\calB}{\mathcal{B}}
\newcommand{\calC}{\mathcal{C}}
\newcommand{\calE}{\mathcal{E}}
\newcommand{\calF}{\mathcal{F}}
\newcommand{\calG}{\mathcal{G}}
\newcommand{\calH}{\mathcal{H}}
\newcommand{\calK}{\mathcal{K}}
\newcommand{\calM}{\mathcal{M}}
\newcommand{\calS}{\mathcal{S}}
\newcommand{\calX}{\mathcal{X}}
\newcommand{\calY}{\mathcal{Y}}

\newcommand{\inparen}[1]{\left{ #1 \right}}
\newcommand{\probtwo}[2]{\mathsf{Pr}_{#1}\left[ #2 \right]}
\newcommand{\set}[1]{\left\{ #1 \right\}}
\newcommand{\ct}{\mathsf{ct}}
\newcommand{\twotimessadv}[1]{\mathsf{2SSAdv}\left[ #1 \right]}



\newlength{\protowidth}
\newcommand{\pprotocol}[5]{
{\begin{figure*}[#3]
\begin{center}
\setlength{\protowidth}{\textwidth}
\addtolength{\protowidth}{-3\intextsep}

\fbox{
        \small
        \hbox{\quad
        \begin{minipage}{\protowidth}
    \begin{center}
    {\bf #1}
    \end{center}
        #5
        \end{minipage}
        \quad}

        }
        
\end{center}
\vspace{-4ex}
\caption{{#4} #2}
\end{figure*}
} }

% the first arg is name of security game
% the second arg is caption
% the third arg is the game description
% the label needs to be included 
\newcommand{\securitygame}[4]{
   \pprotocol{#1}{#2}{ht!}{#3}{#4}
}

\newcommand{\constr}[4]{
   \pprotocol{#1}{#2}{tbh!}{#3}{#4}
}

\begin{document}

\noindent
\large\textbf{Anish Banerjee, Shankh Gupta} \hfill \textbf{Problem Set - 1}   \\
\normalsize COL759: Cryptography \hfill August 2023\\
\hr


\prob{1: CPA with Very Weak Ciphertext Integrity}
\begin{solution}

\end{solution}


\prob{2 : Encryption Scheme with Threshold Decryption}
\begin{solution}
   
\end{solution}

\prob{3 : One-time secure MACs, and Upgrading One-Time MACs to
Many-Time MACs}
\begin{solution}
    
\end{solution}


\prob{4 : CCA Security v/s Authenticated Encryption}
\begin{solution}
    
\end{solution}

\prob{5: Modular Arithmetic and Basic Group Theory}
\begin{solution}
    \begin{enumerate}[(a)]
        \item Since $a$ and $p$ are coprime, by the Extended Euclid's Agorithm:
        $$ ab+py=\gcd(a,p)=1    $$
        Taking modulo p on both sides:
        $$ ab \mod p =1    $$
        Where $b\in\Z_p$ (If not then by the division algorithm $b=qp+b', b'<p$. So, we can replace $b$ with $b'$)
        \vspace{20pt}

        Now suppose there exist $b, b'\in\Z_p$ such that 
        $$ ab=1\mod p \hspace{50pt} ab'=1\mod p    $$
        Then by definition of mod, $p|a(b-b')$. So $b-b'=0$ since $a$ and $b-b'$ will be coprime to $p$. Hence $b$ is unique.

        \item Consider $h(y)=y^2+y$ and $n=6$. For 3 values of $y$ viz. $2,3,5$, we have $h(y)=0\mod 6$. Thus 
        $$|\{y\in\Z_6:y^2+y=0\mod6\}|=3>2$$

        \item Let $a\in\Z_p$ and $r=\ord(a)$. Then $a^r=1\mod p$. By Fermat's Little Theorem:
        $$a^{p-1}=1\mod p$$
        Suppose by the division algorithm, $p-1=rq+s$, $s<r$. Since $a^{p-1}=1\mod p$ and $a^{r}=1\mod p$, $$a^{p-1-rq}=1\mod p$$ and hence $a^{s}=1\mod p$. But since $s<r$, it must be 0.
    \end{enumerate}
\end{solution}

\end{document}
